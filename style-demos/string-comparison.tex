

\documentclass[a5paper,10pt,german]{article}
% \usepackage[nottoc]{tocbibind}
% \setcounter{secnumdepth}{0}
\usepackage{wasysym} % provides symbols
\usepackage{minted} % syntax coloring

\usepackage{cxltx-style-base}
\usepackage{cxltx-style-cjk}
\usepackage{cxltx-style-shadequote}
\usepackage{cxltx-style-equals}

\newfontfamily\frakturFont[Scale=1]{Gotenburg A}
\newcommand{\fraktur}[1]{\frakturFont #1}
\newfontfamily\floralsFont[Scale=3]{IM FELL FLOWERS 1}
\newcommand{\florals}[1]{{\center\floralsFont #1\par}}

\newenvironment{verdict}{{\fraktur Verdict.}—}{\par}
\newenvironment{advice}{{\fraktur Advice.}—}{\par}

\makeatletter
\newcommand*{\miniscule}{\@setfontsize\miniscule\@vpt{6}}
% \newcommand*{\tiny}{\@setfontsize\tiny\@vipt{7}}
% \newcommand*{\scriptsize}{\@setfontsize\scriptsize\@viipt{8}}
\renewcommand*{\large}{\@setfontsize\large\@xipt{13.6}}
\renewcommand*{\Large}{\@setfontsize\Large\@xiipt{14.5}}
\renewcommand*{\LARGE}{\@setfontsize\LARGE\@xivpt{18}}
\renewcommand*{\huge}{\@setfontsize\huge\@xviipt{22}}
\renewcommand*{\Huge}{\@setfontsize\Huge\@xxpt{25}}
\newcommand*{\HUGE}{\@setfontsize\HUGE\@xxvpt{30}}
\makeatother



\usepackage[
  xetex,
  % showframe,
  left=20mm,
  right=20mm,
  top=9mm,
  bottom=12mm,
  headsep=1mm,
  footskip=5mm,
  marginparsep=1mm,
  marginparwidth=14mm,
  headheight=5mm,
  includehead,
  includefoot,
  paperwidth=148mm,
  paperheight=210mm]{geometry}

\usepackage{etoolbox}
\parindent=0mm
\parskip=2.5mm
% \paGaugeSample

% First, let's define some LaTeX and TeX commands with texts we want to compare:
\newcommand{\cmdA}{sometext}
\newcommand{\cmdB}{sometext}
\newcommand{\cmdC}{othertext}
\def\defA{sometext}
\def\defB{sometext}
\def\defC{othertext}

\newcommand{\cmditem}[5]{\textbackslash \texttt{#1}\{\prm{#2}\}\{\prm{#3}\}\{\prm{#4}\}\{\prm{#5}\}}
\newcommand{\cmd}[1]{\textbackslash \texttt{#1}}
\newcommand{\prm}[1]{$\langle$\textit{\thinspace #1\thinspace}$\rangle$}




\begin{document}

\begingroup% AW, Design of Books
  \pagestyle{empty}
  \centering
  % {\Huge \cn{中日韓文}}\\[\baselineskip]
  % {\Huge \cn{字形索引}}\\[\baselineskip]
  ~\vspace{2\baselineskip}\\
  % \florals{IJ}
  \florals{KLM}
  ~\vspace{2\baselineskip}\\

  {\Huge\textit{The}}\\[1\baselineskip]
  {\HUGE\scFont\LaTeX\\[0.25\baselineskip]
  String~Equality\\[0.25\baselineskip]
  Comparison~Primer}\\[1\baselineskip]
  {\Huge\textit{by}}\\[1\baselineskip]
  {\HUGE\cn{慈夆流}}
  % {\HUGE \cn{字面索引}}\\[\baselineskip]
  % {\HUGE \scFont Chinese, Japanese \textit{\&} Korean}\\[0.75\baselineskip]
  % {\HUGE \scFont \textit{Characters}}\\[3\baselineskip]

  \vspace{2\baselineskip}
  % % {Containing \thejzrglyphcount\ characters}\\[8\baselineskip]
  % {\LARGE\textit{Containing}}\\[0.25\baselineskip]
  % {\Huge 9980 characters}\\[0.25\baselineskip]
  % % {\LARGE\textit{grouped}}\\%[0.5\baselineskip]
  % % {\LARGE\textit{on geometric principles}}\\%[0.5\baselineskip]
  % % {\LARGE\textit{by}}\\[0.5\baselineskip]
  % {\LARGE\textit{grouped on geometric principles by}}\\[0.25\baselineskip]
  % {\Huge 657 guides}\\[0.25\baselineskip]
  % {\LARGE\textit{which are divided into}}\\[0.25\baselineskip]
  % {\Huge 119 shapeclasses}\\[0.25\baselineskip]
  % {\LARGE\textit{according to their initial strokes}}\\[0.25\baselineskip]
  \florals{KLM}
  \vfill
  \thispagestyle{empty} % repeated here as per http://tug.org/pipermail/macostex-archives/2010-February/043062.html
  % {\small\sffamily THE PUBLISHER}\par
  \endgroup

% front matter
% \null\newpage\thispagestyle{empty}\null\newpage\thispagestyle{empty}\tableofcontents
\null\newpage\thispagestyle{empty}\tableofcontents

\newpage
\section{Rationale}

While there is a justified interest in the question what it really {\em is}, in essence, that distinguishes a
Turing-complete programming language from, say, mere markup, sane pragmatics know very well that this
not-so elusive bit lies in

\RIGHTarrow~variables,\\
\RIGHTarrow~conditionals (if/then/else\ldots),\\
\RIGHTarrow~loopage (loop, while, break\ldots), and\\
\RIGHTarrow~anonymous and named functions.

All of these four factors are (in principle) part of \TeX. You certainly could do with less {\em in
principle}, but {\em in practice} (read: most of the time), you want your idiom to possess all four factors,
as program sources quickly become obfuscated and convoluted if any one of these is missing or poorly
implemented.

In the present document, we are solely concerned with conditionals and branching, more specifically: how
to write \LaTeX\ code that branches according to a string equality test.

Equality testing is the most fundamental of all tests, so fundamental that a programming language with a
poor implementation of equality testing must be considered a poor programming language.

According to this metric, \TeX\ (and \LaTeX) are poor programming languages indeed. Anyone who has heard
of \TeX's \verb#\ifx# command, be it from the Book or from the Net, and tried to make it work and not
commit a blunder that leads to mysterious, subtle and hard-to-debug errors can testify to this.

Forunately for us, the community that has grown up around \TeX—and especially \LaTeX—has provided the world
of free and open-source software with a plethora of packages that make many things much easier for Joe
the average \LaTeX\ writer.

To the best of my knowledge, easy and straightforward string equality testing has sadly not yet been fully
covered.

In this paper, i want to discuss the merits of the four most popular string equality
commands that float around in forum discussions: these are \verb#\ifdefstrequal#,
\verb#\ifdefstring#, and \verb#\ifstrequal#, all from the \verb#etoolbox# package, and \TeX's `primeval'
(and confusingly named) \verb#\ifx# command.

In the last section, i then propose a new macro by the name of \verb#\eqTextEquals#, which combines the
knowledge gained from documentation \& testing, and makes string equality testing a no-brainer.

\newpage
\section{A Word of Caution}

During the preparation of this document, an Error krept in, where i had forgotten to define macro
\verb#\defC#. Now we all know well that whenever you use an undefined name in your \TeX source, say:

\verb#\unknown#

then the compiler will most certainly opt-out with

\verb#! Undefined control sequence.#\\
\verb#l.73 \unknown#

But not so when you're doing string comparison. Testing two texts for equality requires such deep and
mysterious inner workings of \TeX to be used that testing whether a given name is defined gets swept under
the carpet somehow. So you {\em can} write this:

\verb#\ifdefstrequal{\cmdA}{\unknown}{if branch}{else branch}#

and \verb#\ifdefstrequal# will duly choose the else branch—which is correct, in a way, since an undefined
value can never ever equal any defined value (though in JavaScript, this is not true). Still, this is an
unexpected behavior, as the Principle of Least Surprise would dictate that undefined names {\em always}
make \TeX\ halt with error. So take special care you have all your marbles defined, and no typos in your
string comparison arguments.


\newpage
\section{Setup}

To perform the tests, the following definitions are presupposed:

\verb#\newcommand{\cmdA}{sometext}#\\
\verb#\newcommand{\cmdB}{sometext}#\\
\verb#\newcommand{\cmdC}{othertext}#\\
\verb#\def\defA{sometext}#\\
\verb#\def\defB{sometext}#\\
\verb#\def\defB{othertext}#

All tests are done in pairs that are similar to this one:

\verb#\ifstrequal{\cmdA}{sometext}{OK}{wrong}#\\
\verb#\ifstrequal{\cmdA}{othertext}{wrong}{OK}#

Both clauses of a pair do a string comparison using the command under scrutiny; the first one should execute
the \verb#if# clause, the second one the \verb#else# clause of the statement. If a test succeeds, an
\verb#OK# is displayed; if it fails, \verb#wrong# is printed out. Thus, a given argument pattern is usable
with the given command if two \verb#OK#s are printed behind the command quotations; in case the word
\verb#wrong# appears, the command is not suited for the kind of arguments used in that test pair—be it \TeX\
\verb#\def#s, \LaTeX\ \verb#\newcommand#s, or string literals.

\newpage
\section{ifstrequal}

\cmditem{ifstrequal}{string}{string}{true}{false}

\begin{shadequote}{etoolbox.pdf}
Compares two strings and executes \prm{true} if they are equal, and \prm{false} otherwise. The strings are
not expanded in the test and the comparison is category code agnostic. Control sequence tokens in any of the
\prm{string} arguments will be detokenized and treated as strings. This command is robust.
\end{shadequote}


\verb#\ifstrequal{sometext}{sometext}#:      \ifstrequal{sometext}{sometext}{OK}{wrong}\\
\verb#\ifstrequal{sometext}{othertext}#:     \ifstrequal{sometext}{othertext}{wrong}{OK}

\verb#\ifstrequal{\cmdA}{sometext}#:      \ifstrequal{\cmdA}{sometext}{OK}{wrong}\\
\verb#\ifstrequal{\cmdA}{othertext}#:     \ifstrequal{\cmdA}{othertext}{wrong}{OK}

\verb#\ifstrequal{\defA}{sometext}#:      \ifstrequal{\defA}{sometext}{OK}{wrong}\\
\verb#\ifstrequal{\defA}{othertext}#:     \ifstrequal{\defA}{othertext}{wrong}{OK}

\verb#\ifstrequal{\cmdA}{\cmdB}#:         \ifstrequal{\cmdA}{\cmdB}{OK}{wrong}\\
\verb#\ifstrequal{\cmdA}{\cmdC}#:         \ifstrequal{\cmdA}{\cmdC}{wrong}{OK}

\verb#\ifstrequal{\cmdA}{\defB}#:         \ifstrequal{\cmdA}{\defB}{OK}{wrong}\\
\verb#\ifstrequal{\cmdA}{\defC}#:         \ifstrequal{\cmdA}{\defC}{wrong}{OK}

\verb#\ifstrequal{\defA}{\cmdB}#:         \ifstrequal{\defA}{\cmdB}{OK}{wrong}\\
\verb#\ifstrequal{\defA}{\cmdC}#:         \ifstrequal{\defA}{\cmdC}{wrong}{OK}

\verb#\ifstrequal{\defA}{\defB}#:         \ifstrequal{\defA}{\defB}{OK}{wrong}\\
\verb#\ifstrequal{\defA}{\defC}#:         \ifstrequal{\defA}{\defC}{wrong}{OK}

\begin{verdict}
A jolly useless command as such, as it merely checks whether two literal texts are equal. In many years of
programming, i've virtually {\em never} felt the urge to litter my sources with non-consequential
incantations à la \verb#42 == 42#. Still, it's useful inside macro definitions, as shown in the definition
of \verb#\eqTextEquals#, below.
\end{verdict}

\begin{advice}
You never want to use this one, except when writing a macro yourself.
\end{advice}

\newpage
\section{ifdefstring}


\cmditem{ifdefstring}{command}{string}{true}{false}

\begin{shadequote}{etoolbox.pdf}
Compares the replacement text of a \prm{command} to a \prm{string} and executes \prm{true} if they are
equal, and \prm{false} otherwise. Neither the \prm{command} nor the \prm{string} is expanded in the test and
the comparison is category code agnostic. Control sequence tokens in the \prm{string} argument will be
detokenized and treated as strings. This command is robust. Note that it will only consider the replacement
text of the \prm{command}.
\end{shadequote}


\verb#\ifdefstring{\cmdA}{sometext}#:     \ifdefstring{\cmdA}{sometext}{OK}{wrong}\\
\verb#\ifdefstring{\cmdA}{othertext}#:    \ifdefstring{\cmdA}{othertext}{wrong}{OK}

\verb#\ifdefstring{\defA}{sometext}#:     \ifdefstring{\defA}{sometext}{OK}{wrong}\\
\verb#\ifdefstring{\defA}{othertext}#:    \ifdefstring{\defA}{othertext}{wrong}{OK}

\verb#\ifdefstring{\cmdA}{\cmdB}#:        \ifdefstring{\cmdA}{\cmdB}{OK}{wrong}\\
\verb#\ifdefstring{\cmdA}{\cmdC}#:        \ifdefstring{\cmdA}{\cmdC}{wrong}{OK}

\verb#\ifdefstring{\cmdA}{\defB}#:        \ifdefstring{\cmdA}{\defB}{OK}{wrong}\\
\verb#\ifdefstring{\cmdA}{\defC}#:        \ifdefstring{\cmdA}{\defC}{wrong}{OK}

\verb#\ifdefstring{\defA}{\cmdB}#:        \ifdefstring{\defA}{\cmdB}{OK}{wrong}\\
\verb#\ifdefstring{\defA}{\cmdC}#:        \ifdefstring{\defA}{\cmdC}{wrong}{OK}

\verb#\ifdefstring{\defA}{\defB}#:        \ifdefstring{\defA}{\defB}{OK}{wrong}\\
\verb#\ifdefstring{\defA}{\defC}#:        \ifdefstring{\defA}{\defC}{wrong}{OK}

\begin{verdict}
Use this command to compare a \TeX\ \verb#\def# or \LaTeX\ \verb#\newcommand# expansion with a literal
text. If the second argument is or contains a macro, only the \verb#else# clause will be executed.

\verb#\ifdefstring# should prove useful in a lot of circumstances. The fact that the first argument can be
both a \verb#\def# or a \verb#\newcommand# makes it robust (in the sense that you don't need to know which
command was defined in what way); the fact that the second argument must be a literal helps to keep the
source simple and readable.
\end{verdict}

\begin{advice}
Use \verb#\ifdefstring# especially in cases where a macro is a state-keeping variable that operates on
short keywords.
\end{advice}


\newpage
\section{ifdefstrequal}

\cmditem{ifdefstrequal}{command}{command}{true}{false}

\begin{shadequote}{etoolbox.pdf}
Performs a category code agnostic string comparison of the replacement text of two commands. This command is
similar to \cmd{ifdefstring} except that both arguments to be compared are macros. This command is robust.
\end{shadequote}


\verb#\ifdefstrequal{\cmdA}{sometext}#:   \ifdefstrequal{\cmdA}{sometext}{OK}{wrong}\\
\verb#\ifdefstrequal{\cmdA}{othertext}#:  \ifdefstrequal{\cmdA}{othertext}{wrong}{OK}

\verb#\ifdefstrequal{\defA}{sometext}#:   \ifdefstrequal{\defA}{sometext}{OK}{wrong}\\
\verb#\ifdefstrequal{\defA}{othertext}#:  \ifdefstrequal{\defA}{othertext}{wrong}{OK}

\verb#\ifdefstrequal{\cmdA}{\cmdB}#:      \ifdefstrequal{\cmdA}{\cmdB}{OK}{wrong}\\
\verb#\ifdefstrequal{\cmdA}{\cmdC}#:      \ifdefstrequal{\cmdA}{\cmdC}{wrong}{OK}

\verb#\ifdefstrequal{\cmdA}{\defB}#:      \ifdefstrequal{\cmdA}{\defB}{OK}{wrong}\\
\verb#\ifdefstrequal{\cmdA}{\defC}#:      \ifdefstrequal{\cmdA}{\defC}{wrong}{OK}

\verb#\ifdefstrequal{\defA}{\cmdB}#:      \ifdefstrequal{\defA}{\cmdB}{OK}{wrong}\\
\verb#\ifdefstrequal{\defA}{\cmdC}#:      \ifdefstrequal{\defA}{\cmdC}{wrong}{OK}

\verb#\ifdefstrequal{\defA}{\defB}#:      \ifdefstrequal{\defA}{\defB}{OK}{wrong}\\
\verb#\ifdefstrequal{\defA}{\defC}#:      \ifdefstrequal{\defA}{\defC}{wrong}{OK}

\begin{verdict}
Works as advertised.
\end{verdict}

\begin{advice}
Use \verb#\ifdefstrequal# to ckeck whether two macros have the same string expansion.
\end{advice}

\newpage
\section{ifx}

\cmd{ifx}\prm{command}\prm{command}\prm{true}\prm{false}\cmd{fi}

\begin{shadequote}{http://en.wikibooks.org/wiki/TeX/ifx}
With the \verb#\ifx# command you can perform conditional compilation by testing for macro equivalence.
It does not expand the given macros. The two macros are considered equal if\\
\RIGHTarrow~both are macros and \\
\RIGHTarrow~the first level expansion is equal and \\
\RIGHTarrow~they have the same state with regards to \verb#\long# and \verb#\outer#.
\end{shadequote}


\verb#\ifx\cmdA\cmdB#\ifx\cmdA\cmdB\ OK    \else\ wrong \fi\\
\verb#\ifx\cmdA\cmdC#\ifx\cmdA\cmdC\ wrong \else\ OK    \fi

\verb#\ifx\cmdA\defB#\ifx\cmdA\defB\ OK    \else\ wrong \fi\\
\verb#\ifx\cmdA\defC#\ifx\cmdA\defC\ wrong \else\ OK    \fi

\verb#\ifx\defA\cmdB#\ifx\defA\cmdB\ OK    \else\ wrong \fi\\
\verb#\ifx\defA\cmdC#\ifx\defA\cmdC\ wrong \else\ OK    \fi

\verb#\ifx\defA\defB#\ifx\defA\defB\ OK    \else\ wrong \fi\\
\verb#\ifx\defA\defC#\ifx\defA\defC\ wrong \else\ OK    \fi


\begin{verdict}
This macro's description hurts my feelings. Seriously. According to the tests, it would {\em appear} to work
correctly when both
arguments are either \TeX\ \verb#\def#s or else \LaTeX\ \verb#\newcommand#s, but be aware that in the
general case you will {\em not} know for sure whether a given \verb#\x# was defined in whatever which way
(unless you did it yourself a short while ago or you perused the sources). The restriction regarding
\verb#\long# and \verb#\outer# makes \verb#\ifx# even more brittle and ultimately useless for anyone
but the most hard-core \TeX nocrats.
\end{verdict}

\begin{advice}
Use with care and only after testing whether the command works for your special case. Consider to use
\verb#\ifdefstrequal# instead, as that one offers fewer opportunities to write buggy code.
Also, the syntax of \verb#\ifdefstrequal# is more in keeping with \LaTeX conventions.
\end{advice}

\newpage
\section{eqTextEquals}

\cmditem{eqTextEquals}{valueOne}{valueTwo}{true}{false}

\begin{shadequote}{}
Uses the \verb#\ifdefmacro# command from the \verb#\etoolbox# package to test whether \prm{valueOne} and
\prm{valueTwo} are macros or literals; based on this knowledge, it decides whether to use
\verb#\ifdefstrequal#, \verb#\ifdefstring#, or \verb#\ifstrequal# to perform a string comparison.

\end{shadequote}


\verb#\eqTextEquals{sometext}{sometext}#:      \eqTextEquals{sometext}{sometext}{OK}{wrong}\\
\verb#\eqTextEquals{sometext}{othertext}#:     \eqTextEquals{sometext}{othertext}{wrong}{OK}

\verb#\eqTextEquals{\cmdA}{sometext}#:   \eqTextEquals{\cmdA}{sometext}{OK}{wrong}\\
\verb#\eqTextEquals{\cmdA}{othertext}#:  \eqTextEquals{\cmdA}{othertext}{wrong}{OK}

\verb#\eqTextEquals{\defA}{sometext}#:   \eqTextEquals{\defA}{sometext}{OK}{wrong}\\
\verb#\eqTextEquals{\defA}{othertext}#:  \eqTextEquals{\defA}{othertext}{wrong}{OK}

\verb#\eqTextEquals{\cmdA}{\cmdB}#:      \eqTextEquals{\cmdA}{\cmdB}{OK}{wrong}\\
\verb#\eqTextEquals{\cmdA}{\cmdC}#:      \eqTextEquals{\cmdA}{\cmdC}{wrong}{OK}

\verb#\eqTextEquals{\cmdA}{\defB}#:      \eqTextEquals{\cmdA}{\defB}{OK}{wrong}\\
\verb#\eqTextEquals{\cmdA}{\defC}#:      \eqTextEquals{\cmdA}{\defC}{wrong}{OK}

\verb#\eqTextEquals{\defA}{\cmdB}#:      \eqTextEquals{\defA}{\cmdB}{OK}{wrong}\\
\verb#\eqTextEquals{\defA}{\cmdC}#:      \eqTextEquals{\defA}{\cmdC}{wrong}{OK}

\verb#\eqTextEquals{\defA}{\defB}#:      \eqTextEquals{\defA}{\defB}{OK}{wrong}\\
\verb#\eqTextEquals{\defA}{\defC}#:      \eqTextEquals{\defA}{\defC}{wrong}{OK}

% \newdimen\lengthA
% \newdimen\lengthB
% \newdimen\lengthC
% \lengthA=42mm
% \lengthB=42mm
% \lengthC=108mm

% \verb#\eqTextEquals{\lengthA}{\lengthB}#:      \eqTextEquals{\lengthA}{\lengthB}{OK}{wrong}\\
% \verb#\eqTextEquals{\lengthA}{\lengthC}#:      \eqTextEquals{\lengthA}{\lengthC}{wrong}{OK}


\begin{verdict}
Works as advertised.
\end{verdict}

\begin{advice}
Use \verb#eqTextEquals# to replace all the other commands discussed in this document—it's much simpler, more
versatile, less surprising, and puts less strain onto your name memory.
\end{advice}

\newpage
\subsection*{Implementation}

The \verb#\eqTextEquals# macro has a fairly simple implementation, {\em viz.}:

\begin{minted}{latex}
\newcommand{\eqTextEquals}[4]{%

  \ifdefmacro{#1}{%
    % ...................................................
    % #1 is a macro
    \ifdefmacro{#2}{%
      % .................................................
      % #1 and #2 are macros
        \ifdefstrequal{#1}{#2}{#3}{#4}
      }{%
        % ...............................................
        % #1 is a macro, #2 is a literal
        \ifdefstring{#1}{#2}{#3}{#4}
      }%
  }{%
    % ...................................................
    % #1 is a literal
    \ifdefmacro{#2}{%
      % .................................................
      % #1 is a literal, #2 is a macro
        \ifdefstring{#2}{#1}{#3}{#4}
      }{%
        % ...............................................
        % #1 and #2 are literals
        \ifstrequal{#1}{#2}{#3}{#4}
      }%
    }%
  }%
\end{minted}



\end{document}






% \ifx

% \ifdefstrequal
% \ifdefstring
% \ifstrequal

